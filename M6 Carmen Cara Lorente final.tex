\documentclass[10pt,a4paper]{article}
\usepackage[utf8]{inputenc}
\usepackage{amsmath}
\usepackage{enumitem}
\usepackage{amsfonts}
\usepackage{amssymb}
\usepackage{graphicx}
\begin{document}
\title{Derechos Humanos: Concepto y clasificación}
\author{Carmen Cara Lorente}
\date{}
\maketitle
\providecommand{\keywords}[1]{\textbf{\textit{Keywords:}} #1}
\begin{abstract}
URL de GITBUB: https://github.com/carmencara92/proyecto\_final\\

El concepto de los Derechos Humanos actualmente está regido gracias a una de las organizaciones internacionales más importantes del mundo: Naciones Unidas (ONU). A raíz de este concepto aportado por la ONU, la academia ha ido desarrollando su concepto, crítica y clasificación de distintas maneras. De esta manera, los Derechos Humanos se pueden clasificar de distintas formas: existe la clasificación por generaciones y la clasificación según el tipo de derecho humano en cuestión. En este artículo, vamos a estudiar dichas clasificaciones con ejemplos.

\begin{figure}
\centering
\includegraphics{unhumanrightscouncil}
\caption{Consejo de los Derechos Humanos de la ONU}
\end{figure}
\end{abstract}
\keywords{Derechos Humanos, Clasificación, Naciones Unidas}
\newpage
\section{Introducción: el concepto de Derechos Humanos.}
El concepto de Derecho Humano tiene orígenes remotos en la Filosofía del Derecho y en el Derecho Natural, y en concreto en la Antigua Grecia así como los orígenes del cristianismo. De esta manera, una sencilla fórmula que se puede emplear sobre la idea con la que se empezaron a desarrollar los Derechos Humanos podría ser esta:
\begin{figure}[h]
\[
    \binom {Justicia} = \frac{Valores+Virtudes}{Igualdad}= {Legalidad}
\]
\caption{Autor: Jesús Orlando Morales Ortiz, [5]}
\end{figure}

Con respecto a la temática de este ensayo, se ha escogido la definición de Derechos Humanos aportada por las Naciones Unidas: \\
"Los derechos humanos son derechos inherentes a todos los seres humanos, sin distinción alguna de raza, sexo, nacionalidad, origen étnico, lengua, religión o cualquier otra condición. Entre los derechos humanos se incluyen el derecho a la vida y a la libertad; a no estar sometido ni a esclavitud ni a torturas; a la libertad de opinión y de expresión; a la educación y al trabajo, entre otros muchos. Estos derechos corresponden a todas las personas, sin discriminación alguna." [4]\\
Esta definición ha sido seleccionada debido a la gran relevancia en el campo de los Derechos Humanos de esta organización internacional. La definición aportada por Naciones Unidas se basa en la Declaración Universal de los Derechos Humanos de 1948 y uno de sus principales objetivos es la protección de los Derechos Humanos a nivel internacional[4]. Para lograr dicho objetivo, Naciones Unidas ha establecido diversos mecanismos de protección internacional de los Derechos Humanos. Entre ellos, se pueden nombrar tratados, procedimientos especiales sancionadores de protección, el Consejo de Derechos Humanos de Naciones Unidas, etc.\\
Debido a la importancia institucional de Naciones Unidas en el ámbito de los Derechos Humanos, su definición es usada en la literatura científica [2] y como guía para la sociedad civil y otros actores internacionales.\\
\\

 
\newpage
\section{Estado del Arte.}
Naciones Unidas aporta la noción sobre la clasificación de los Derechos Humanos a partir de los tipos de tratados creados por esta misma organización, a la par que ofrece una lista exhaustiva de los mismos [4]. Sin embargo,  se debe hacer una revisión de la literatura legal para ver cómo se ha venido abordando el tema de los Derechos Humanos desde la doctrina jurídico-política, puesto que la de Naciones Unidas no es la única clasificación de los Derechos Humanos existente.\\
\\
-Clasificación basada en el sistema de Naciones Unidas.\\

Por una parte, Bantekas y Oette dan un repaso a la historia del Derecho Internacional de los Derechos Humanos y examinan la creación de distintos organismos internacionales, incluyendo Naciones Unidas. Establecen los mecanismos jurídicos y políticos de la ONU como el marco de referencia a tener en cuenta a la hora de afrontar el estudio de los Derechos Humanos. Asimismo, tienen en cuenta los sistemas regionales de protección internacional, tales como la Unión Europea o el sistema Interamericano. A la hora de examinar los Derechos Humanos de manera exhaustiva, optan por una clasificación basada en los tipos de derecho o en su contenido. De esta manera, distinguen entre Derechos Civiles  y Políticos de los Derechos Económicos, Sociales y Culturales. [2]\\
De igual manera, Malcom Shaw, repasa los principios del Derecho Internacional de los Derechos Humanos desde la base de las Naciones Unidas. Así, su clasificación de los Derechos Humanos se establece a partir de los convenios de esta organización internacional, a saber: Derechos Civiles y Políticos y Derechos Económicos, Sociales y Culturales [6].\\
\\
-Clasificación basada en el nivel obligatorio o vinculante.\\
\\
Otra clasificación adoptada por la doctrina se basa en la distinción entre el "hard law" o "soft law" de los Derechos Humanos. Una distinción que da cierta importancia jerárquica entre unos derechos de otros. Tal distinción es usada por autores tales como Forsythe. Dicho autor también explora el papel de otros actores en el ámbito de los Derechos Humanos, tales como la sociedad civil o los medios de comunicación [3].\\
\\
-Clasificación temporal.\\
\\
Otra clasificación se refiere a cuándo se han ido desarrollando los derechos. En este sentido habría Derechos Humanos de distintas generaciones (primera, segunda y tercera). Los de primera generación surgen con la Revolución Francesa, los de segunda generación con la Revolución Industrial y los de tercera generación se producen entre el Siglo XX y XXI.Dicha clasificación es la principal usada por Aguilar Cuevas, la cual califica esta separación entre Derechos Humanos como "la más conocida". [1]
\newpage
\section{Discusión: ¿Qué clasificación usar?}

Una vez revisadas las distintas clasificaciones usadas por la doctrina científica, cabe cuestionarse cuál de todas resulta ser la más adecuada.\\

Es conveniente usar una clasificación de los Derechos Humanos que no use una jerarquía, puesto que todos los Derechos Humanos son igualmente importantes y todos constituyen la base para una vida humana digna, por lo tanto, el sistema entre "hard law" y "soft law" debería ser descartado.\\
De igual manera, una clasificación temporal de los mismos resultaría impráctica, puesto que todos esos derechos han sido recopilados históricamente y son aplicados en la actualidad, sin importar demasiado sus orígenes históricos y sí si se garantizan a los seres humanos a día de hoy. Tal clasificación tendría sentido a la hora de entender los orígenes de tales derechos.\\
Sin embargo, la extendida clasificación adoptada por Naciones Unidas es ampliamente usada por la doctrina y se basa en una clasificación adoptada por el cuerpo internacional más importante del mundo en términos de protección de Derechos Humanos. Por lo tanto, este artículo considera tal clasificación como la más idónea.\\
Cabe preguntarse ahora, ¿Qué derechos son Civiles y Políticos y cuáles son los Económicos Sociales y Culturales? La siguiente tabla lo ilustra bien:

\begin{figure}[h]
\begin{center}
\begin{tabular}{ |c|c|  }
 \hline
 \textbf{Derechos C y P} & \textbf{Derechos E S y C}\\ 
  \hline
 Derecho a la vida & Derecho a la salud\\  
 Derecho a la no tortura & Derecho al agua\\
 Derecho a la libertad & Derecho a la educación\\
 Derecho a un juicio justo & Derecho a la comida\\
 Derecho a la no desaparición forzada & Derecho a la educación\\
 \hline 
\end{tabular}
\end{center}

\caption{Elaboración propia}
\end{figure}
\newpage
\section{Bibliografía}

\begin{thebibliography}{9}
\bibitem{aguilar} 
Magdalena Aguilar Cuevas.
\textit{Las Tres Generaciones de los Derechos Humanos}.
\\\texttt{file:///C:/Users/Usuario/AppData/Local/Temp/5117-4516-1-PB.pdf
}

\bibitem{Bantekas and Oette} 
Ilias Bantekas and Lutz Oette. 
\textit{International Human Rights. Law and Practice}. 
Cambridge University Press, 2017.
 
\bibitem{Forsythe} 
David P. Forshyte. 
\textit{Human Rights in International Relations}.
Cambridge University Press, 2018.
\bibitem{naciones unidas} 
Naciones Unidas.
\textit{Declaración Universal de los Derechos Humanos, 1948}.

\bibitem{orlando} 
Jesús Orlando Morales Ortíz.
\textit{Jurismatematiologia: Estudio y tratado del derecho desde el punto de vista matemático}.
\\\texttt{https://www.monografias.com/trabajos91/jurismatematiologia-estudio-y-tratado-del-derecho-punto-vista-matematico/jurismatematiologia-estudio-y-tratado-del-derecho-punto-vista-matematico.shtml
}
 
\bibitem{shaw} 
Malcom N. Shaw.
\textit{International Law}.
Cambridge University Press, 2014.
\end{thebibliography}

\end{document}
